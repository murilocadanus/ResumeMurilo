\documentclass{resume}

\usepackage[left=0.75in,top=0.6in,right=0.75in,bottom=0.6in]{geometry}
\usepackage[utf8]{inputenc}
\usepackage{hyperref}

\name{Murilo C. da Costa}
\address{ Nilo Peçanha, 3803 \\ Curitiba, PR - 82120-440 - Brésil  }
\address{ skype: murilocadanus \\ murilocadanus@gmail.com }

\def\nameskip{\bigskip}
\def\sectionskip{\medskip}

\begin{document}

  \begin{rSection}{Formation}
    {\bf PUC-PR, Curitiba - PR - Brésil} \hfill {\em Mai 2013} \\ 
    { Spécialisation en Informatique - Programmeur de Jeux Vidéo } \\

    {\bf Université Positivo, Curitiba - PR - Brésil} \hfill {\em Déc. 2007} \\ 
    { Baccalauréat en Informatique } \\
  \end{rSection}
  
  \begin{rSection}{Connaissances linguistiques}
    \begin{tabular}{ @{} >{\bfseries}l @{\hspace{6ex}} l }
      Portugais & Langue Maternelle \\
      Anglais & Avancé \\
      Français & Avancé \\
    \end{tabular}
  \end{rSection}

  \begin{rSection}{Expérience Profissionnelle}
    \begin{rSubsection}{Global Game Jam, 7DRL Game Jam}{Janv. 2013, Janv. 2014}{Architecte et programmeur}{Curitiba, PR}
    \item Projet Reapers (\url{https://github.com/ggj/reapers/commits?author=murilocadanus}) – C'est un jeu de plateforme 2D. J'ai développé utilisant \textbf{C++} avec l'IDE \textbf{QTCreator}, le moteur \textbf{Seed} et le \textbf{libRocket} API pour l'UI.  J'ai utilisé le \textbf{Git} pour la gestion de versions du source code. \\
    \item Projet Optimalus (\url{https://github.com/ggj/optimalus/commits?author=murilocadanus}) – C'est un jeu de style Rogue Like. J'ai utilisé \textbf{C++} avec l'IDE \textbf{QTCreator}, le moteur \textbf{Seed} et le \textbf{libRocket} API pour l'UI.  J'ai utilisé le \textbf{Git} pour la gestion de versions du source code. \\
    \item Projet QuakeRL (\url{https://github.com/7drl/QuakeRL/commits?author=murilocadanus})– J'ai développé le module de pathfind utilisant \textbf{C++} avec l'IDE \textbf{QTCreator}, le moteur \textbf{Seed} et le \textbf{libRocket} API pour l'UI.  J'ai utilisé le \textbf{Git} pour la gestion de versions du source code.
    \end{rSubsection}
    \begin{rSubsection}{Winterlabs Technologies}{Déc. 2010 - 
Actuelle}{Architecte et programmeur}{Curitiba, PR}
    \item Projet CORE RUNNER (\url{https://itunes.apple.com/us/app/core-runner/id821017105}) - Le Core Runner est un jeu 3D procédural, où le joueur doit courir et collecter cubes pour ponctuer. Je l'ai développé en \textbf{C\#} avec \textbf{MonoDevelop}, \textbf{Unity3D} et \textbf{NGUI} plugin. J'ai integré avec le Apple IAP pour vendre les items en jeu et Apple Game Centre pour le highscores. J'ai utilisé le \textbf{Git} pour la gestion du code source. \\
    \item Projet STAP – C'est un simulateur de tir virtuel pour les armes portables. J'ai développé le module qui contrôle la piste de tir utilisant \textbf{C\#} avec \textbf{VisualStudio}, J'ai utilisé l'\textbf{Unity3D} pour rendre les scènes du stand de tir et \textbf{NGUI} plugin pour l'UI, \textbf{Crystal Reports} pour développer les rapports sur les sessions de tir et \textbf{MySQL} comme sauvegarde.  J'ai utilisé le \textbf{Mercurial} pour la gestion de versions du source code. \\
    \item Projet SALVA CARGAS (\url{http://salvacargas.herokuapp.com}) - Le Salva Cargas est un jeu éducatif avec l'objective de développer le raisonnement mathématique. Le jouer doit utiliser un canon et deux bateaux pour récupérer des charges qui volent sur une île. Je l'ai développé en \textbf{C\#} avec \textbf{MonoDevelop}, \textbf{Unity3D} et \textbf{NGUI} plugin. J'ai utilisé le \textbf{Git} pour la gestion du code source. Le jeu a été compilé pour fonctionner avec la technologie Flash. \\
    \item Projet ETERNAL LEGENDS (\url{http://youtu.be/LW1yF8D-CKU}) - Le Eternal Legends est un jeu vidéo de carte en 3D pour le Mac OS X et le Windows. Le joueur doit utiliser ses cartes pour détruire l'adversaire. Je l’ai développé en \textbf{C++} avec \textbf{Xcode} et \textbf{VisualStudio}, l’architecture utilise \textbf{SDL} pour contrôler les actions d’entrée et le son. Pour le rendu j’ai utilisé \textbf{OpenGL}. Les agents de IA ont été développés et exposés comme scripts \textbf{Lua} pour faciliter la modularisation. J’ai utilisé \textbf{Git} pour la gestion du code source. \\
    \item Projet MR FALL (\url{http://itunes.apple.com/br/app/mrfall/id420213660}) – Mr Fall est un jeu de plateforme en 2D pour \textbf{iOS}. Le jeu a 12 niveaux répartis dans trois différents paysages: montagne, forêt et cave. L'objectif est de recueillir tous les diamants dans le niveau et trouver la sortie. Je l’ai développé en \textbf{C\#} avec \textbf{MonoDevelop}, \textbf{Unity3D}, j'ai utilisé comme moteur physique le \textbf{PhysX}, j'ai intégré avec Mobclix, AdMob et Millenial Media network et \textbf{Git} pour  contrôler le code source. \\
    \item Projet SKY FIGHTERS (\url{http://itunes.apple.com/us/app/skyfighters-hd/id419328364}) - Sky Fighters est un jeu de carte et stratégie pour \textbf{iOS}. Les joueurs utilisent trois cartes qui représentent trois avions. L'objectif du jeu est de détruire la base adverse ou les avions. Je l’ai développé en \textbf{Obj-C} avec \textbf{Xcode}, framework \textbf{cocos2d}, design patterns, Model Node Controller, Singleton, AbstractFactory en architecture, intégré avec PlayHeaven et Game Center d'Apple avec les réalisations et le score leaderboard classé. J’ai utilisé \textbf{Git} pour la gestion de code source. \\
    \item Projet RESTA1 (\url{http://itunes.apple.com/br/app/resta1/id322697159}) – Jeu de plateau classique pour \textbf{iOS} dans lequel le joueur doit éliminer le nombre maximum de pièces du plateau. Je l’ai développé en \textbf{Obj-C} avec \textbf{Xcode}, \textbf{cocos2d} et intégrée avec AdMob. J’ai utilisé \textbf{Git} pour la gestion du code source.
    \end{rSubsection}
    \begin{rSubsection}{GVT Telecom}{Août 2008 - Déc. 2010}{Analyste et programmeur}{Curitiba, PR}
    \item Projet "Nova Arquitetura OSS” – Modification de nouveaux équipements pour augmenter la vitesse du produit ADSL. J'ai développé une nouvelle interface de communication avec le nouvel équipement ADSL, j'ai utilisé \textbf{JEE 1.5} avec \textbf{Oracle 10g} et le serveur \textbf{BEA Weblogic application server}, \textbf{BEA Aqualogic Service Bus}, \textbf{EJB3}, \textbf{JPA}, \textbf{JAX-WS 2.0}, \textbf{JAXB} et \textbf{Eclipse}.\\
    \item Projet PD “Modificação Portal Integra” – Modification pour fournir de nouvelles informations à la clientèle. J'ai développé l'interface  utilisant des \textbf{portlets} avec \textbf{BPM} et l'architecture \textbf{SOA} utilisant \textbf{Savvion}, j'ai utilisé le serveur d'applications \textbf{BEA Weblogic Portal}, \textbf{BEA Aqualogic Service Bus}.\\
    \item Projet “PJ 759 – Nova Regulamentação de Call Centers”. Projet pour répondre à la nouvelle réglementation des centres d'appel. J‘ai spécifié le IT Solution avec \textbf{Enterprise Architec} et j'ai développé les \textbf{Stored Procedures} et \textbf{Triggers} en \textbf{PL/SQL}, \textbf{Oracle} comme \textbf{SGBDR} et \textbf{JSP}, \textbf{JSTL}, \textbf{Struts}, \textbf{EJB 2.1} sous \textbf{Oracle IAS} serveur d'application.\\
    \item Projet “PJ 679 - TT-Retail”. Projet avec l’objectif de manoeuvre des stations de base existantes. J’ai utilisé \textbf{BEA WebLogic Workshop 8}, \textbf{BEA AquaLogic Service Bus}, \textbf{WebLogic Integration (WLI)}, \textbf{JMS}, \textbf{Oracle AQ}, \textbf{MDB} et \textbf{BEA Weblogic Application Server 8}.\\
    %\item Projet PJ 771 - WiseTools. Web services qui exposent les données contenues de la base du TBS/OSS. J’ai utilisé \textbf{JEE 1.5} avec \textbf{Oracle 10g} et \textbf{BEA Weblogic serveur d'application}, \textbf{BEA AquaLogic Service Bus}, \textbf{EJB3}, \textbf{JPA}, \textbf{JAX-WS 2.0}, \textbf{JAXB}, \textbf{Eclipse}, \textbf{Mylyn}. J'ai utilisé une méthode de developpement agile (\textbf{SCRUM}) avec l'outil \textbf{JIRA} et \textbf{Green Hopper} (agile plugin pour burndown chart).\\
    %\item Projet “PJ 702 – Serviços de Gerência”. C’est un système SLA pour gérer les données et la qualité du transfert de voix des clients directs. J’ai utilisé les technologies \textbf{Servlet}, \textbf{JSP}, \textbf{JSTL}, \textbf{AJAX}, \textbf{EJB-doclet}, \textbf{EJB 2.1}, \textbf{Spring}, \textbf{JMS}, \textbf{MDB}, \textbf{Web Services}, \textbf{JasperReports}, \textbf{JFreeChart}, \textbf{ANT} et les design patterns, \textbf{Session Facade}, \textbf{DTO}, \textbf{Business Object}, \textbf{Service Locator}. J'ai utilisé le \textbf{BEA WebLogic Workshop 9.2} comme IDE, le \textbf{BEA Weblogic 9.2} comme serveur d'applications et \textbf{BEA AquaLogic Service Bus} comme \textbf{ESB}.\\
    %\item Projet “Arquitetura Integração Contínua OSS” – Definition de l’architecture de développement d’intégration continue pour les projets de l'équipe OSS-GVT. J'ai utilisé les technologies: \textbf{SVN} pour contrôle de code source, \textbf{Maven} pour compiler le code source, le programme \textbf{NEXUS} pour contrôle d'artefacts, \textbf{BAMBOO} pour gérer et planifier les actions de code source, \textbf{JIRA} comme Issue Tracker, \textbf{JUnit} pour tester le code, \textbf{Selenium} pour faire des tests unitaires automatiques et les plugins \textbf{Checkstyle}, \textbf{PMD}, \textbf{STATSVN}, \textbf{Cobertura}, \textbf{FindBugs} pour faire des rapports sur la qualité du code.
    \end{rSubsection}
    \begin{rSubsection}{HSBC Banque}{Janv. 2006 - Mai. 2008}{Analyste et programmeur}{Curitiba, PR}
    \item Projet “HOB-PWS” - Système développé pour les clients sans compte bancaire; modules Intranet et Internet. J’ai utilisé (Intranet): pattern \textbf{MVC} (\textbf{ITA Framework} / \textbf{MVC2}), intégration avec \textbf{CICS} \textbf{IBM MQSeries} les technologies (Internet): \textbf{Servlets}, \textbf{XML} et \textbf{XSL}. \textbf{WSAD}, \textbf{Websphere 5.1}.\\
    \item Projet “CDCI”. Système de simulation de financement direct pour le client. J’ai fait la spécification de l’application utilisant le \textbf{IBM RSA}. J’ai développé dans le modèle \textbf{MVC} avec \textbf{Eclipse-RCP}, \textbf{SWT}, la sérialisation d'objets comme couche de sauvegarde.\\
    \item Projet “HOB-PG-TIT”. Système de paiement de titres utilisés par les clients qui accèdent au site internet de la banque HSBC. Je l’ai développé avec \textbf{Servlets}, \textbf{XML} et \textbf{XSL}, Architecture \textbf{SOA} avec le \textbf{BPM} utilisant \textbf{RTP} (processeur de transactions à distance), l'intégration avec l'héritage (\textbf{CICS}), \textbf{IBM MQSeries}, \textbf{WSAD} (WebSphere Application Developer), \textbf{WebSphere 5.1}.\\
    %\item Projet “GRH - Système de ressources humaines. Je l’ai développé avec le modèle \textbf{MVC} (\textbf{Framework ITA} / \textbf{MVC2}) avec \textbf{JDBC} et \textbf{Sybase}. \textbf{WSAD}; \textbf{Websphere 5.1}
    \end{rSubsection}
  \end{rSection}

  \begin{rSection}{Certifications}
    {\bf Sun Microsystems, Inc., Curitiba - PR - Brésil} \hfill {\em Mai 2005} \\ 
    {SUN CERTIFIED PROGRAMMER FOR THE JAVA 2 PLATFORM 1.4} \\

    {\bf Sun Microsystems, Inc., Curitiba - PR - Brésil} \hfill {\em Mai 2008} \\ 
    {SUN CERTIFIED WEB COMPONENT DEVELOPER FOR THE JAVA 2 PLATFORM ENTERPRISE EDITION 1.5} \\
  \end{rSection}
  
\end{document}
