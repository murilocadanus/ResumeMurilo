\documentclass{resume}

\usepackage[left=0.75in,top=0.6in,right=0.75in,bottom=0.6in]{geometry}
\usepackage[utf8]{inputenc}
\usepackage{hyperref}

\name{Murilo C. da Costa}
\address{ Nilo Peçanha, 3803 \\ Curitiba, PR - 82120-440 - Brasil  }
\address{ skype: murilocadanus \\ murilocadanus@gmail.com }

\def\nameskip{\bigskip}
\def\sectionskip{\medskip}

\begin{document}

  \begin{rSection}{Formação}
    {\bf PUC-PR, Curitiba - PR - Brasil} \hfill {\em Mai. 2013} \\ 
    { Especialização em Ciência da Computação - Desenvolvimento de Jogos Digitais } \\

    {\bf Universidade Positivo, Curitiba - PR - Brasil} \hfill {\em Dez. 2007} \\ 
    { Bacherelado em Informática } \\
  \end{rSection}
  
  \begin{rSection}{Idiomas}
    \begin{tabular}{ @{} >{\bfseries}l @{\hspace{6ex}} l }
      Português & Nativo \\
      Inglês & Avançado \\
      Francês & Avançado \\
    \end{tabular}
  \end{rSection}

  \begin{rSection}{Experiência profissional}
    \begin{rSubsection}{EBTS}{Fév. 2013 - Presente}{Arquiteto e programador}{Curitiba, PR}
    \item Projeto STAP – Simulador de tiro virtual para armas portáteis. Atuei no desenvolvimento do estande de tiro virtual usando a linguagem \textbf{C\#} com o \textbf{VisualStudio} e também com o \textbf{Unity3D} para renderizar as cenas, para a UI foi usado o plugin \textbf{NGUI}, \textbf{Crystal Reports} para desenvolver os relatórios referentes as sessões de tiro e \textbf{MySQL} como persistência. Usei também o SCV \textbf{Mercurial} para gerênciar as versões geradas e o código fonte. \\
    \item Projeto STAP 180$^\circ$ Video – Simulador de tiro virtual de 180$^\circ$ para armas portáteis que usa video. Implementei a reprodução de videos em 180$^\circ$. Desenvolvi em \textbf{C++} com a API do \textbf{DirectShow} para criar um pipeline de reprodução de Audio e Video. Utilizei o \textbf{VisualStudio} e o \textbf{Mercurial} para gerênciar as versões geradas e o código fonte.
    \end{rSubsection}
    \begin{rSubsection}{Winterlabs Technologies}{Dez. 2010 - Jan. 2013}{Arquiteto e programador}{Curitiba, PR}
    \item Projeto SALVA CARGAS (\url{http://salvacargas.herokuapp.com}) - O Salva Cargas é um jogo educacional com o objetivo de desenvolver o raciocínio matemático. Eu desenvolvi usando \textbf{C\#} com o \textbf{MonoDevelop}, \textbf{Unity3D} e o plugin \textbf{NGUI}. Para a gestão do código fonte e versões usei o \textbf{Git}. O jogo foi compilado para funcionar através do Flash player ou Unity player, usando um web browser. \\
    \item Projeto ETERNAL LEGENDS (\url{http://youtu.be/LW1yF8D-CKU}) - O Eternal Legends é um jogo em 3D de cartas para Mac OS X e Windows. Desenvolvi usando \textbf{C++} com o \textbf{Xcode} e \textbf{VisualStudio}, na arquitetura usei o \textbf{SDL} para controle de entrada e saída e reprodução de som. Para rederização usei a API \textbf{OpenGL}. Os agentes de IA foram desenvolvidos e expostos como scripts \textbf{Lua} para facilitar a modularização. Utilizei o \textbf{Git} para gestão do código fonte e de versões. \\
    \item Projeto MR FALL (\url{http://itunes.apple.com/br/app/mrfall/id420213660}) – Mr Fall é um jogo de plataforma 2D para \textbf{iOS}. O jogo tem 12 niveis dividido em 3 diferentes mundos: montanha, floresta e caverna. Desenvolvi usando \textbf{C\#} com o \textbf{MonoDevelop}, \textbf{Unity3D}, usei como motor de física o \textbf{PhysX}, foi integrado com as redes Mobclix, AdMob et Millenial Media network. Usei o \textbf{Git} para controle de versão. \\
    \item Projeto SKY FIGHTERS (\url{http://itunes.apple.com/us/app/skyfighters-hd/id419328364}) - Sky Fighters é um jogo de estratégia usando cartas para \textbf{iOS}. Os jogadores usam 3 cartas que represemtam 3 aviões e o objetivo é destruir o esquadrão do oponente. Desenvolvi em \textbf{Obj-C} com o \textbf{Xcode}, usando o framework \textbf{cocos2d}, os design patterns, Model Node Controller, Singleton, AbstractFactory. Realizei a integração com as redes PlayHeaven et Game Center da Apple com ranking de pontos e troféus de conquistas. Usei o \textbf{Git} para gestão de versão e código fonte. \\
    \item Projeto RESTA1 (\url{http://itunes.apple.com/br/app/resta1/id322697159}) – Jogo de tabuleiro para \textbf{iOS} sendo que o jogador deve eliminar o maior número de peças possíveis. Desenvolvi em \textbf{Obj-C} com o \textbf{Xcode}, \textbf{cocos2d} e integrei com a rede AdMob. usei o \textbf{Git} para gerenciar versões e o código fonte.
    \end{rSubsection}
    \begin{rSubsection}{GVT Telecom}{Ago. 2008 - Dez. 2010}{Analista e programador}{Curitiba, PR}
    \item Projeto "Nova Arquitetura OSS” – Modificação de novos equipamentos para aumentar a velocidade do produto ADSL. Desenvolvi uma nova interface de comunicação com o equipamento ADSL, usei \textbf{JEE 1.5} com \textbf{Oracle 10g}, \textbf{BEA Weblogic application server}, \textbf{BEA Aqualogic Service Bus}, \textbf{EJB3}, \textbf{JPA}, \textbf{JAX-WS 2.0}, \textbf{JAXB} e \textbf{Eclipse}.\\
    \item Projeto PD “Modificação Portal Integra” – Modificação para fornecer novas informações aos clientes. Desenvolvi a interface usando \textbf{portlets}, usei \textbf{BPM} em uma arquitetura \textbf{SOA} utilizando \textbf{Savvion}, usei o application server \textbf{BEA Weblogic Portal} e também o \textbf{BEA Aqualogic Service Bus}.\\
    \item Projeto “PJ 759 – Nova Regulamentação de Call Centers”. Projeto referente a nova regulamentação de call centers definido pela Anatel. Desenvolvi a especificação usando o \textbf{Enterprise Architec} e desenvolvi as \textbf{Stored Procedures} e \textbf{Triggers} em \textbf{PL/SQL}, usando \textbf{Oracle} como \textbf{SGBDR} e \textbf{JSP}, \textbf{JSTL}, \textbf{Struts}, \textbf{EJB 2.1} sobre o application server \textbf{Oracle IAS}.\\
    \item Projeto “PJ 679 - TT-Retail”. Projeto com o objetivo de manobrar estações de equipamentos de comunicação existentes. Desenvolvi usando o \textbf{BEA WebLogic Workshop 8}, \textbf{BEA Aqualogic Service Bus}, \textbf{WebLogic Integration (WLI)}, \textbf{JMS}, \textbf{Oracle AQ}, \textbf{MDB} e \textbf{BEA Weblogic Application Server 8}.\\
    \item Projeto PJ 771 - WiseTools. Web services que expõem os dados de equipamentos de comunicação. Desenvolvi usando \textbf{JEE 1.5} com \textbf{Oracle 10g} e com o application server \textbf{BEA Weblogic Server}, \textbf{BEA AquaLogic Service Bus}, \textbf{EJB3}, \textbf{JPA}, \textbf{JAX-WS 2.0}, \textbf{JAXB}, \textbf{Eclipse}, \textbf{Mylyn}. Desenvolvi usando um método ágil (\textbf{SCRUM}) com a ferramenta \textbf{JIRA} e o plugin \textbf{Green Hopper} (plugin para burndown chart).\\
    \item Projeto “PJ 702 – Serviços de Gerência”. Sistema SLA para genrenciar as informações de qualidade de tranferência de dados e voz dos clientes. Desenvolvi usando as tecnologias \textbf{Servlet}, \textbf{JSP}, \textbf{JSTL}, \textbf{AJAX}, \textbf{EJB-doclet}, \textbf{EJB 2.1}, \textbf{Spring}, \textbf{JMS}, \textbf{MDB}, \textbf{Web Services}, \textbf{JasperReports}, \textbf{JFreeChart}, \textbf{ANT} e os design patterns, \textbf{Session Facade}, \textbf{DTO}, \textbf{Business Object}, \textbf{Service Locator}. Usei o \textbf{BEA WebLogic Workshop 9.2} como IDE, o \textbf{BEA Weblogic 9.2} como application server e o \textbf{BEA AquaLogic Service Bus} como \textbf{ESB}.\\
    \item Projeto “Arquitetura Integração Contínua OSS” – Definição da arquitetura e desenvolvimento de integração continua para os projetos da equipe OSS-GVT. Utilizei as tecnologias: \textbf{SVN} para controle do código fonte, \textbf{Maven} para compilação do código, o plugin \textbf{NEXUS} para controle de artefatos, \textbf{BAMBOO} para gerenciar e agendar as ações a serem realizadas com os códigos fonte, \textbf{JIRA} como Issue Tracker, \textbf{JUnit} para testes do código, \textbf{Selenium} para fazer testes unitários automáticos e os plugins \textbf{Checkstyle}, \textbf{PMD}, \textbf{STATSVN}, \textbf{Cobertura}, \textbf{FindBugs} para fazer os relatórios sobre a qualidade do código fonte.
    \end{rSubsection}
    \begin{rSubsection}{HSBC}{Jan. 2006 - Mai. 2008}{Analista e programador}{Curitiba, PR}
    \item Projeto HOB-PWS - Sistema desenvolvido para os clientes sem conta bancária; modulos Intranet e Internet. Eu usei no sistema Intranet o design pattern \textbf{MVC} usando os frameworks \textbf{ITA} / \textbf{MVC2}, integrei com o legado \textbf{CICS} usando filas \textbf{IBM MQSeries}. No sistema Internet usei \textbf{Servlets}, \textbf{XML} e \textbf{XSL}, IDE \textbf{WSAD}. O deploy foi realizado no application server \textbf{Websphere 5.1}.\\
    \item Projeto CDCI. Sistema de simulação de financiamento. Desenvolvi a especificação da aplicação usando o \textbf{IBM RSA}. Desenvolvi usando o design pattern \textbf{MVC} com o \textbf{Eclipse-RCP}, \textbf{SWT} e a persistência foi feita através de serialização de objetos.\\
    \item Projeto HOB-PG-TIT. Sistema de pagamento de títulos, usado por clientes do internet banking do HSBC. Desenvolvi usando \textbf{Servlets}, \textbf{XML} et \textbf{XSL}, a arquitetura é \textbf{SOA} com \textbf{BPM} utilizando \textbf{RTP}, a integração com o legado usa (\textbf{CICS}), e \textbf{IBM MQSeries}, usei a IDE \textbf{WSAD}, e foi feito o deploy no application server \textbf{WebSphere 5.1}.\\
    \item Projeto GRH - Sistema de Recursos Humanos. Desenvolvi com o padrão \textbf{MVC} usando os frameworks \textbf{ITA} / \textbf{MVC2} usando \textbf{JDBC} e \textbf{Sybase}. Usei a IDE \textbf{WSAD} e o deploy foi feito no application server \textbf{Websphere 5.1}
    \end{rSubsection}
  \end{rSection}

  \begin{rSection}{Certificações}
    {\bf Sun Microsystems, Inc., Curitiba - PR - Brasil} \hfill {\em Mai 2005} \\ 
    {SUN CERTIFIED PROGRAMMER FOR THE JAVA 2 PLATFORM 1.4} \\

    {\bf Sun Microsystems, Inc., Curitiba - PR - Brasil} \hfill {\em Mai 2008} \\ 
    {SUN CERTIFIED WEB COMPONENT DEVELOPER FOR THE JAVA 2 PLATFORM ENTERPRISE EDITION 1.5} \\
  \end{rSection}
  
\end{document}
